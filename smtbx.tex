%------------------------------------------------------------------------------
% Template file for the submission of papers to IUCr journals in LaTeX2e
% using the iucr document class
% Copyright 1999-2012 International Union of Crystallography
% Version 1.5 (7 March 2012)
%------------------------------------------------------------------------------

\documentclass{iucr}

\usepackage{cctbx_notations}
\usepackage{amsmath}

\journalcode{A}           % Indicate the journal to which submitted
                                  %   A - Acta Crystallographica Section A
                                  %   B - Acta Crystallographica Section B
                                  %   C - Acta Crystallographica Section C
                                  %   D - Acta Crystallographica Section D
                                  %   E - Acta Crystallographica Section E
                                  %   F - Acta Crystallographica Section F
                                  %   J - Journal of Applied Crystallography
                                  %   S - Journal of Synchrotron Radiation

\begin{document}

\newcommand{\olexrefine}{olex2.refine}

\title{The Olex 2 refinement engine}
\shorttitle{\olexrefine}  % Abbreviated title for use in running heads

     % Authors' names and addresses. Use \cauthor for the main (contact) author.
     % Use \author for all other authors. Use \aff for authors' affiliations.
     % Use lower-case letters in square brackets to link authors to their
     % affiliations; if there is only one affiliation address, remove the [a].

\newcommand{\brukerfr}{Bruker AXS-SAS, 4 Allée Lorentz, 77447 Marne-la-Vallée cedex 2, \country{France}}
\newcommand{\durhamchem}{Chemistry Department, Durham University, South Road, Durham, DH1~3LE, \country{UK}}
\newcommand{\olexsys}{OlexSys Ltd, \durhamchem}
\newcommand{\cci}{CCI, Lawrence Berkeley Laboratory, 1 Cyclotron Road, BLDG 64R0121, Berkeley, CA 94720-8235, \country{USA}}

\cauthor{Luc}{Bourhis}{luc\_j\_bourhis@mac.com}{\brukerfr}
\author{Oleg}{Dolomanov}
\author{Richard}{Gildea}
\author{Judith A K}{Howard}
\author{Horst}{Puschmann}

\aff{\durhamchem}

\shortauthor{Bourhis, Dolomanov, Gildea, Howard and Puschmann} % abbreviated author list for use in running heads

\keyword{small molecules}\keyword{refinement}\keyword{constraints}\keyword{restraints}\keyword{least-squares}

\maketitle

\begin{synopsis}
Supply a synopsis of the paper for inclusion in the Table of Contents.
\end{synopsis}

\begin{abstract}
Abstract goes here.
\end{abstract}

\section{Constraints}

Constraints may be conceptualised mainly in two manners, that we will first illustrate on a simple example.

\appendix
\section{Computation of structure factors and their gradient}

The formulae discussed in this appendix have been known for nearly a century and have been implemented in all known refinement programs. However it seems to the authors that, during the last decades, the computation for a give Miller index $h$ of $F_c(h)$ and its gradient $\grad{F_c(h)}$ with respect to crystallographic parameters has very rarely been presented in all the minute details necessary to implement those formulae in a program, which justifies this appendix in our humble opinion.


\ack{Acknowledgements}

This work has been founded by the EPSRC grant ``Age Concern: Crystallographic Software for the Future" (EP/C536274/1) from the British government. We wish to thank David Watkin with whom we shared this grant for his immense contribution to our understanding of crystallographic refinement. We also wish to thank all the contributors to the CCTBX library which served as a foundation and model for our work, especially its main author Ralf W. Grosse-Kunstleve whose support has been invaluable. Finally we wish to thanks Pr. George Sheldrick for many an invaluable discussion.

\referencelist[smtbx] % References using the BibTeX database smtbx.bib

\end{document}
